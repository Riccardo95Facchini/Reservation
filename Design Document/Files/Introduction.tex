\subsection{Purpose}
This document aims to detail the design of the software and of the architecture regarding the application Reservation.
To do so it will be taken a more detailed approach for the description of each component and the overall architecture of the system.

\subsection{Scope}
Reservation is an appointment management application, designed to help the users with their daily routine by keeping track for them of their next appointments in registered shops and avoiding the hustle of queueing or trying to reach the shop managers on the phone.\\
The same application can be used by \textbf{customers} and \textbf{shop owners} to manage the reservations, for the former it will help with the actual reservation process and act as an agenda by keeping track of the next appointments taken, while for the latter it will display the next customers that reserved a spot while giving only minimum information to preserve the customer's privacy.\\
Both kinds of users have access to chat functionalities that allow them to communicate if needed and a back-end handled system of notifications that will notify them when a new reservation has been made (receivable for shops only as they can't make a reservation themselves) or a new chat message is available.\\
Customers are also able to leave a review score from 1 to 5 (stars) and in future update that value if they wish so.

\subsection{Definitions, Acronyms, Abbreviation}
\begin{itemize}
\item DB: DataBase
\item DBMS: DataBase Management System
\item GPS: Global Position System
\item UID: Unique Identifier
\end{itemize}

\subsection{Document Structure}
This document is structured has:
\begin{enumerate}
	\item \textbf{Introduction}, it provides an overview of the entire document.
	\item \textbf{Architectural Design}, it describes different views of components and their interaction.
		\item \textbf{Algorithm Design}, it describes the main algorithm and query methods.
	\item \textbf{User Interface Design}, it provides an overview about the aspect of the user interfaces of the system.
	\item \textbf{Implementation, Integration and Testing}, it describes the orders in which the components are implemented and the order of the integration with some testing.
	\item \textbf{Appendix}, it contains software used.
\end{enumerate}



